\section{Introduction}
Throughout this tutorial, \fbox{boxes} will be used in the following two cases:
\begin{itemize}
\item to clarify the precedences of symbols when formulae become too large. \\
e.g. \fbox{$\Gvdash \fbox{$\lam x M$}~:~\fbox{$(y:A)\to B$}
\Leftarrow \fbox{$\lam x u$}$}.
\item to distinguish type theory terms from natural language text. \\
e.g. ``we combine a term \fbox{$a$} with a term \fbox{$b$} to get a term \fbox{$a~b$}''.
\end{itemize}
\subsection{Target Audience}
This tutorial assumes familiarity with the following:
\begin{itemize}
\item Dependent type theory concepts, such as formation rules,
introduction rules, eliminators, etc., and functional programming.
\item Programming and theorem proving in a proof assistant based on dependent type theories.
\item The ability to translate (simple) typing rules into a type-checking procedures
and combine them into an algorithm.
\item Basic understanding of De Morgan cubical type theory~\cite{CCHM,CHM}
(hereafter as \CTT{}), including the interval type $\II$, the path type,
the idea of representing $n$-dimensional cubes using terms with interval variables in it,
and the De Morgan operators on $\II$.
\end{itemize}
This tutorial will not treat substitution formally --
variable names are assumed to respect capture-avoiding substitution.
In the implementations, any binding representation that works for untyped
$\lambda$-calculus should work for the type theory introduced in this tutorial.
\begin{notation}\label{not:pre}
We will prefer using $x, y, z$ for variables
and other Latin letters like $u,v,a,b,c,A,B,C$ for terms
(preferably uppercase for types and lowercase for terms).

Instead of the more traditional \fbox{$\lambda x.b$},
the notation for $\lambda$-abstraction is \fbox{$\lam x b$} 
following the style of the Arend language.
We will also use the conventional shorthand
\fbox{$(x:A)~(y:B)\to C$} for nested $\Pi$-types.

Substitution is denoted by $u[v/x]$.
One may think of this notation as ``fractional multiplication'' $u\times \frac v x$,
where the denominator $x$ is cancelled out from $u$ and \textit{replaced} with the numerator $v$.
Other authors may use $u[x\mapsto v]$, $u[x:=v]$, $[v/x]u$, etc.

Definitional equality (a.k.a. judgmental equality) is denoted $u\equiv v$.
\end{notation}
\begin{notation}\label{not:syntax-def}
We will write \fbox{$u, A::=$} for syntax definition of terms,
and define the syntax of \CTT{} by extending the syntax of Martin-L\"of type theory
with a few term former at a time, instaed of putting everything together
in a single, unified BNF grammar.
The typing rules will be introduced similarly.

We will extend the BNF grammar with the \textit{list} operator like \fbox{$\overline{(x_i:A_i)}\to B$},
which means that the string below the line can be repeated one or more times,
optionally indexed by a subscript such as $i$.
\end{notation}
Some quick warm-ups:
\begin{exercise}
Translate the following typing rules into an algorithmic description:
\begin{mathpar}
\inferrule{\Gamma,x:A\vdash b:B[x/y]}{\Gvdash \fbox{$\lam x b$} : (y:A)\to B} \and
\inferrule{\Gvdash u : (x:A)\to B \\ \Gvdash v:A}{\Gvdash u~v:B[v/x]}
\end{mathpar}
Which of these is an introduction rule, and which is an elimination rule?
\end{exercise}
\begin{exercise}\label{ex:concat-sym}
Consider path concatenation and symmetry in \CTT{}:
\begin{align*}
\concat&:(p:a=b)\to(q:b=c)\to{a=c}\\
\sym&:(p:a=b)\to{b=a}
\end{align*}
Define both using \hcomp{} on the following squares,
preferably in a cubical programming language:
% https://q.uiver.app/?q=WzAsOCxbMCwwLCJhIl0sWzEsMCwiYyJdLFswLDEsImEiXSxbMSwxLCJiIl0sWzIsMSwiYSJdLFsyLDAsImIiXSxbMywxLCJhIl0sWzMsMCwiYSJdLFswLDEsIlxcY29uY2F0KHAsIHEpIiwwLHsic3R5bGUiOnsiYm9keSI6eyJuYW1lIjoiZGFzaGVkIn19fV0sWzIsMCwiIiwwLHsibGV2ZWwiOjIsInN0eWxlIjp7ImhlYWQiOnsibmFtZSI6Im5vbmUifX19XSxbMiwzLCJwIiwyXSxbMywxLCJxIiwyXSxbNCw1LCJwIl0sWzQsNiwiIiwyLHsibGV2ZWwiOjIsInN0eWxlIjp7ImhlYWQiOnsibmFtZSI6Im5vbmUifX19XSxbNiw3LCIiLDIseyJsZXZlbCI6Miwic3R5bGUiOnsiaGVhZCI6eyJuYW1lIjoibm9uZSJ9fX1dLFs1LDcsIlxcc3ltKHApIiwwLHsic3R5bGUiOnsiYm9keSI6eyJuYW1lIjoiZGFzaGVkIn19fV1d
\[\begin{tikzcd}
	a & c & b & a \\
	a & b & a & a
	\arrow["{\concat(p, q)}", dashed, from=1-1, to=1-2]
	\arrow[Rightarrow, no head, from=2-1, to=1-1]
	\arrow["p"', from=2-1, to=2-2]
	\arrow["q"', from=2-2, to=1-2]
	\arrow["p", from=2-3, to=1-3]
	\arrow[Rightarrow, no head, from=2-3, to=2-4]
	\arrow[Rightarrow, no head, from=2-4, to=1-4]
	\arrow["{\sym(p)}", dashed, from=1-3, to=1-4]
\end{tikzcd}\]
\end{exercise}
\subsection{Motivation}
This tutorial is intended to help readers to get more familiar with how
\CTT{} works under the hood, what difficulties it is having in implementations,
what it can already do, and what it cannot do yet.

\CTT{} is a type theory evolved from a model using Kan cubical sets~\cite{CubicalSets},
which uses sophisticated homotopy theory. Computer scientists, on the other hand,
usually do not have relevant courses taught in their undergraduate program.
However, it is also the computer scientists who are supposed to implement \CTT{} as
programming languages. This tutorial tries to help those who did not study homotopy
theory, but have learned about the informal concepts of \CTT{}
(like what is written in~\cite{NCTT}) and wish to learn the implementation details of \CTT.

\CTT{} extends Martin-L\"{o}f type theory with a huge amount of new constructions,
especially the typing rules are written in a very compact way (like in~\cite{HCompPDF}).
This tutorial aims to discuss them from an algorithmic perspective,
and hopefully to inspire more people to implement cubical type theory,
to fuse these ideas into other work, or just to worship these brilliant ideas.

This tutorial is a by-product of an experiment in implementing \CTT, called \GuestName,
a project created to encourage a particular person to learn \CTT.
The story ends up in the worst way: the person did not learn \CTT, and instead created
a new project to encourage the author of \GuestName{} to learn extensional type theory.

\section{Type checking cubes}\label{sec:tyck-cube}
This section introduces the notion of \textit{partial elements} and
motivates typing judgments with cofibrations in the context.

\subsection{The interval $\II$ and contexts}\label{sub:interval}
In \CTT, we have the interval type:
\[\vdash \isType\II\quad \vdash \lcon:\II \quad \vdash \rcon:\II\]
The interval type and its products are used to represent dimensions
(ignoring the De Morgan structures for now).
\begin{example}\label{ex:interval-in-ctx}
Suppose \fbox{$x:\II\vdash\isType{A}$} and \fbox{$x:\II\vdash u:A$}.
From a semantical or a topological perspective, we can say:
\begin{itemize}
\item $A$ is a (type) line between $A[\lcon/x]$ and $A[\rcon/x]$.
\item $u$ is a (term) line between $u[\lcon/x]$ and $u[\rcon/x]$.
\item The type of a line is a line, so the type of $u$ is $A$.
\end{itemize}
Then, because \emph{typing relations are preserved by substitution},
the following typing relations hold:
\begin{mathpar}
\inferrule{}{u[\rcon/x]:A[\rcon/x]}\and
\inferrule{}{u[\lcon/x]:A[\lcon/x]}
\end{mathpar}
We may visualize the fact as:
% https://q.uiver.app/?q=WzAsNixbMCwwLCJ1W1xcbGNvbi9pXSJdLFswLDEsInVbXFxyY29uL2ldIl0sWzIsMCwiQVtcXGxjb24vaV0iXSxbMiwxLCJBW1xcbGNvbi9pXSJdLFsxLDBdLFsxLDFdLFswLDEsInUiXSxbMiwzLCJBIl0sWzQsNSwiOiIsMSx7InN0eWxlIjp7ImJvZHkiOnsibmFtZSI6Im5vbmUifSwiaGVhZCI6eyJuYW1lIjoibm9uZSJ9fX1dXQ==
\[\begin{tikzcd}
	{u[\lcon/x]} & {} & {A[\lcon/x]} \\
	{u[\rcon/x]} & {} & {A[\lcon/x]}
	\arrow["u", from=1-1, to=2-1]
	\arrow["A", from=1-3, to=2-3]
	\arrow["{:}"{description}, draw=none, from=1-2, to=2-2]
\end{tikzcd}\]
\end{example}
From~\cref{ex:interval-in-ctx} we motivate the following notational
convention for contexts in \CTT, as in~\cref{not:ctx}.
\begin{notation}\label{not:ctx}
Typing judgments are written as \fbox{$\PGvdash \isType{A}$} and \fbox{$\PGvdash u:A$},
where \fbox{$\Psi;\Gamma$} is the usual \textit{context} in type theories,
with variables classified into two groups: if a variable has type $\II$,
it goes to $\Psi$, otherwise it goes to $\Gamma$.
This convention is borrowed from~\cite{ABCFHL}.
\end{notation}
Note that~\cref{not:ctx} does not imply that contexts has to be classified
in the implementations. The \GuestName{} type checker mix intervals
and other bindings in a unified context, just like usual dependent type checkers.
\begin{remark}\label{rem:line-ori}
Consider \fbox{$x:\II\vdash u:A$} and \fbox{$\vdash v:\II\to A$}.
Usually both are referred to as a \textit{line}, but they are very different.
Suppose we weaken the context with $y:\II$ to be a $2$-dimensional space,
in which $u$ exists as a line:
\carloCTikZ{\carloXy
\node (0) at (0 , 0) {\textbullet} ;
\node (1) at (1 , 0) {\textbullet} ;
\draw[->] (0) -- (1) node [midway, above] {$u$};}
Note that the orientation of $u$ is fixed to be horizontal.
However, for $v$, we can apply either $x$ or $y$ to get a line oriented differently:
\carloCTikZ{\carloXy
\node (0) at (0 , 0) {\textbullet} ;
\node (1) at (1 , 0) {\textbullet} ;
\node (2) at (0 , 1) {\textbullet} ;
\draw[->] (0) -- (1) node [midway, above] {$v~x$};
\draw[->] (0) -- (2) node [midway, left] {$v~y$};}
So, interval application may also be thought of as
\textit{placing an $n$-dimensional cube at the given orientation}.
\end{remark}
\begin{remark}\label{rem:sq-ori}
Unlike lines as discussed in~\cref{rem:line-ori}, squares are much more flexible.
Consider \fbox{$\vdash u:\II\to\II\to A$} in a $2$-dimensional context,
there are already two different ways to place it:
\carloCTikZ{\carloXy
\carloSqBullets
\fill [pattern color=lightgray,pattern=horizontal lines] (0,0) rectangle (1,1) ;
\node (c) at (0.5, 0.5) {$u~x~y$} ;
\shiftTikZ{1.5,0}{
\carloSqBullets
\fill [pattern color=lightgray,pattern=vertical lines] (0,0) rectangle (1,1) ;
\node[rotate=90,xscale=-1] (c) at (0.5, 0.5) {$u~y~x$} ;
}}
Note that these two placements are symmetric with respect to the diagonal.

In case of contexts and cubes of higher dimensions,
the situations are much more complicated.
For example, with one more dimension $z:\II$,
we may place $u$ in three orientations:
\carloCTikZ{\carloXyz
\refcube{TopUXY}
\shiftTikZ{0,0.3}{\refcube{FrontUXY}}
\shiftTikZ{1.3,0.2}{\refcube{LeftUXY}}
}
Note that all of them can also be reflected by their diagonals.
\end{remark}

\subsection{Partial elements}\label{sub:partial}
In \CTT{}, the idea that \textit{open shapes can be filled}
is the core concept that makes terms in type theory space-like,
and to do that, a \textit{composition operation} is added to \CTT{} as a structure.

The motivation is that with only the interval type and Martin-L\"of type theory,
we may not be able to describe every cube that makes geometric or topological sense.
For example, the squares in~\cref{ex:concat-sym} make perfect sense in
geometry or topology, but in \CTT{}, they have to be constructed using the composition operation.
The composition operation takes a description of some parts of an $n$-dimensional cube
(an \textit{incomplete} cube, or a \textit{partial} cube) and completes it.
To describe the input of composition, we introduce partial elements.
\begin{terminology}
We follow the terminology in~\cite{CubicalAgda}.
In~\cite{CCHM}, partial elements are called \textit{systems}.
\end{terminology}
When we write \fbox{$x:\II,y:\II\vdash u:A$},
we are describing the following 2-dimensional cube, which is a square:
\carloCTikZ{\carloXy
\carloSqBullets
\fill [pattern color=lightgray,pattern=north west lines] (0,0) rectangle (1,1) ;
\node (center) at (0.5, 0.5) {$u:A$} ;}
For simplicity we assume \fbox{$\vdash \isType{A}$}, say, $A$ does not depend on $x$ or $y$.

Suppose we want to use the composition operation to create
a square \fbox{$x:\II,y:\II\vdash u:A$}
such that its top-left corner is a point $a:A$, and the line on its
right-hand side is a line $y:\II \vdash v:A$:
\carloCTikZ{\carloXy
\foreach \y in {0,1} \node (1\y) at (1 , \y) {\textbullet} ;
\node (00) at (0, 0) {$a$};
\node (01) at (0, 1) {\textbullet};
\draw (10) -- node [right] {$v$} (11); %
% \fill [pattern color=lightgray,pattern=north west lines] (0,0) rectangle (1,1) ;
% \node (center) at (0.5, 0.5) {$u:A$} ;
}

Translating that into type theory,
the goal is to construct a term $u$ such that the following holds:
\begin{align*}
u[\lcon/x, \lcon/y]&\equiv a\\
u[\rcon/x]&\equiv v
\end{align*}
The construction will be discussed in later sections,
and for now we focus on how to describe these partial boundaries
(also known as \textit{configurations} of a cube).
We introduce a straightforward syntax called \textit{partial elements}
for these cubes, e.g. the above partial element is written as:
\[\LRbbar{\begin{array}{rc}
  x=\lcon \land y=\rcon&\mapsto a\\
  x=\rcon &\mapsto v
\end{array}}\]
To define this formally, we need to define the syntax of the left-hand-side of $\mapsto$.
They are called \textit{cofibrations} in \CTT.
\begin{terminology}
In~\cite{CCHM}, cofibrations are called \textit{face restrictions},
which is more (geometrically) intuitive but also longer.
\end{terminology}
The syntax of a cofibration is defined to be a disjunction normal form
(a disjunction list of conjunctions) of face \textit{conditions}
like \fbox{$x=\lcon$} or \fbox{$x=\rcon$}, as in~\cref{fig:cofib} (recall~\cref{not:syntax-def}).
\begin{figure}[h!]
\[\begin{array}{rll}
  \cond ::= & x=\lcon \mid x=\rcon & \text{condition} \\
  \conj ::= & \cond~\overline{\land~\cond} & \text{conjunction} \\
  \disj ::= & \bot \mid \top \mid \conj~\overline{\lor~\conj} & \text{disjunction}
\end{array}\]
\caption{Syntax of cofibrations}\label{fig:cofib}
\end{figure}

The typing rules for cofibrations is also straightforward, as in~\cref{fig:tyck-cofib}.
\begin{figure}[h!]
\begin{mathpar}
\inferrule{(x:\II) \in \Psi}{\Psi\vdash \isCond{x=\lcon}} \and
\inferrule{(x:\II) \in \Psi}{\Psi\vdash \isCond{x=\rcon}}\and
\inferrule{\forall i. (\Psi\vdash \isCond{\cond_i})}{\Psi\vdash \bigwedge\nolimits_{i} \cond_i}\and
\inferrule{\forall i. (\Psi\vdash \conj_i)}{\Psi\vdash \bigvee\nolimits_{i} \conj_i}\and
\inferrule{}{\Psi\vdash \top} \and
\inferrule{}{\Psi\vdash \bot}
\end{mathpar}
\caption{Typing rules of cofibrations}\label{fig:tyck-cofib}
\end{figure}

The meaning of cofibrations is simple. Suppose we are in a $3$-dimensional context,
which means there are $3$ intervals in the context, i.e. \fbox{$\Psi:=x:\II,y:\II,z:\II$}.
Then:
\begin{itemize}
\item As mentioned before, the term \fbox{$\Psi\vdash u:A$} corresponds to
(the \textit{filling} of) a $3$-dimensional cube:
\carloCTikZ{\carloXyz \refcube{Empty}}
\item A $\cond$ specifies a $2$-dimensional cube (a square) face in $\Psi$, e.g. \fbox{$x=\lcon$}
corresponds to the following square:
\carloCTikZ{\carloXyz \refcube{XEquivL}}
\item A $\conj$ specifies any $n$-cube (for $n\leq 2$) in $\Psi$, e.g.
\fbox{$x=\lcon\land y=\rcon$} specifies a line $v$,
and \fbox{$x=\lcon\land y=\rcon\land z=\lcon$} specifies a point $\star$:
\carloCTikZ{\carloXyz \refcube{XYLR}}
\item A $\disj$ talks about several $n$-cubes (for $n\leq 2$) in $\Psi$ at the same time, e.g.
\fbox{$x=\lcon\lor x=\rcon$} corresponds to the following two squares:
\carloCTikZ{\carloXyz \refcube{LRFaces}}
\item A $\bot$ cofibration is called the \textit{absurd} cofibration,
which specifies nothing.
\item A $\top$ cofibration is called the \textit{truth} cofibration,
which specifies everything.
\end{itemize}
Using cofibrations, we define the syntax of partial elements
by extending the syntax for terms in~\cref{fig:parEl}.
We define a special kind of partial elements $\lrbbar u$,
which is a partial element with a single face under the truth cofibration,
called \textit{trivial} partial elements.
These partial elements are actually not ``partial'', but ``full''
in the sense that \textit{every} face is assigned with a single term.
\begin{figure}[h!]
\[\begin{array}{rl}
    \face::=&\overline{\conj\mapsto u}\\
    u,A::=&\lrbbar{\face} \mid \lrbbar{u} \mid \cdots~\text{(other term formers, recall~\cref{not:syntax-def})}
\end{array}\]
\caption{Syntax of partial elements}\label{fig:parEl}
\end{figure}

Note that we also need to define the type of partial elements
(hereafter as \textit{partial types}),
and the type needs to contain the following two pieces of information:
\begin{itemize}
\item The faces being specified, $\disj$.
\item The type of the faces, $A$.
\end{itemize}
Thus we directly define the syntax of partial types as in~\cref{fig:parTy}:
\begin{figure}[h!]
\[u,A::= \Partial{\disj}{A} \mid \cdots~\text{(see~\cref{fig:parEl})}\]
\caption{Syntax of partial types}\label{fig:parTy}
\end{figure}

There are several advantages to arrange the cofibrations
as disjunction-normal forms:
\begin{itemize}
\item We put $\conj$ to the left-hand-side of $\mapsto$,
so every clause in a partial element specifies a single face.
\item It is easy to get the type of a partial element:
since each clause specifies a face using $\conj$,
their disjunction is the $\disj$ in the corresponding partial type.
\end{itemize}
It remains to derive the evaluation and typing rules for partial elements.

\subsection{Reducing partial elements by cofibrations}\label{sub:red-cofib}
We reduce well-typed (we will define well-typedness of
partial elements later in~\cref{sub:tyck-cofib}) partial elements by
iterating their face clauses. Consider the following face clause
with no conjunction:
\[x=\lcon\mapsto u\]
We may substitute the variable $x$ with three possible terms of type $\II$:
\begin{itemize}
\item Another variable $y$. In this case, we simply replace $x$ with $y$
and the face becomes $y=\lcon\mapsto u$.
\item $\lcon$, i.e. take the face that $x=\lcon$ in this partial element.
Then, evidently, we get the face $u$, so the partial element should reduce
to the trivial cofibration $\lrbbar u$, with other faces ignored.
In this case, we say that the face is \textit{satisfied}.
\item $\rcon$, i.e. take the face that $x=\rcon$,
which is unspecified by this face, so we drop this particular face and
proceed with the rest of the faces.
In this case, we say that the face is \textit{contradicted}.
\end{itemize}
With the presence of conjunctions, we iterate through each $\cond$,
drop the conditions that are satisfied, and drop the faces
if one of their conditions is contradicted.

There are two more special cases in a conjunction $\conj$:
\begin{itemize}
\item It can be self-contradictory.
This happens when for a variable $x$, we have both $x=\lcon\in\conj$
and $x=\rcon\in\conj$. In this case, we also remove the face.
\item It can contain duplicated information.
For example, we may write $x=\lcon\land x=\lcon$.
It is encouraged to deduplicate these conditions for spatial efficiency.
In this tutorial, we assume deduplication of cofibrations everywhere.
\end{itemize}
By that we may claim the following:
\begin{prop}\label{prop:unique}
In a $\conj$ cofibration, every variable appears uniquely in a $\cond$.
\end{prop}

Substitution may also change partial types \fbox{$\Partial\disj A$}.
The rules are essentially the same as partial elements,
but in case of a face is satisfied, we reduce
the partial type into the annotated type $A$.

\subsection{Type checking under cofibrations}\label{sub:tyck-cofib}
We start from an example of type checking a partial element.
\begin{example}\label{ex:parTyck}
Consider $\Psi:=x:\II, y:\II,z:\II$ and the following:
\begin{mathpar}
\inferrule{}{\Gvdash\isType{A}} \and
\inferrule{}{\Gvdash u:\II\to\II\to A}\and
\inferrule{}{\Gvdash v:\II\to\II\to A}
\end{mathpar}
We want to construct the following partial element (recall~\cref{rem:sq-ori}):
\carloCTikZ{\carloXyz \refcube{BBFaces}}
Recall~\cref{fig:parEl}, we may directly translate that into:
\[\LRbbar{\begin{array}{rll}
z&=\rcon&\mapsto u~x~y\\
y&=\lcon&\mapsto v~x~z
\end{array}}\]
Note that they share the same line \fbox{$z=\rcon\land y=\lcon$}
(obtained by taking the conjunction of both cofibrations).
This line is represented as a cofibration, and we can also use it as
a \textit{substitution}, i.e. $[\rcon/z,\lcon/y]$.
Substituting with this line corresponds to the operation of
taking this line from a cube (that has this line).
Since both faces have this line, we may represent the shared line
by substituting either of them:
\begin{align*}
(u~x~y)[\rcon/z,\lcon/y]&\implies u~x~\lcon\\
(v~x~z)[\rcon/z,\lcon/y]&\implies v~x~\rcon
\end{align*}
However, these two lines are actually the same line,
so they have to be definitionally the same, i.e. $u~x~\lcon\equiv v~x~\rcon$.
In this case, we say that the given two faces \textit{agree}.
For a well-defined partial element, every pair of faces should agree.
The typing rule ends up like this:
\begin{mathpar}
\inferrule{\Hint{\Gvdash\isType{A}}\\
\PGvdash u~x~y:A\\
\PGvdash v~x~z:A\\\\
\PGvdash u~x~\lcon\equiv v~x~\rcon:A}{\PGvdash\LRbbar{\begin{array}{rll}
z&=\rcon&\mapsto u~x~y\\
y&=\lcon&\mapsto v~x~z
\end{array}}:\Partial{(z=\rcon\lor y=\lcon)}A}
\end{mathpar}
\end{example}

We generalize~\cref{ex:parTyck} to an arbitrary partial element (let $i\in I$ for some index set $I$):
\[\lrbbar{\overline{\conj_i\mapsto u_i}}\]
We need to find an algorithm that makes sure every pair of faces agree.
For two faces $\conj_i\mapsto u_i$ and $\conj_j\mapsto u_j$,
from~\cref{ex:parTyck} we know that they overlap at $\conj_i\land\conj_j$
(in case they do not overlap, this cofibration is self-contradictory).
The rule becomes something like:
\begin{mathpar}
\inferrule{\Hint{\Gvdash\isType{A}}\\
\forall i\in I.(\Psi\vdash \conj_i)\\
\forall i\in I.(\PGvdash u_i:A)\\\\
\forall i,j\in I.(\PGvdash[\text{overlap-check}])}
{\PGvdash\lrbbar{\overline{\conj_i\mapsto u_i}}
:\Partial{\left(\bigvee\nolimits_{i\in I} \conj_i\right)}A}
\end{mathpar}
In~\cref{ex:parTyck}, the overlap-check step is a conversion check
between a \textit{substituted version} of the two faces $u_i$ and $u_j$.
For convenience, we introduce the typing judgment as in~\cref{fig:conv-cofib}.
\begin{figure}[h!]
\[\Psi;\Gamma,\conj\vdash u\equiv v:A\]
\caption{Conversion check under a conjunction cofibration}\label{fig:conv-cofib}
\end{figure}

To implement this judgment, we convert $\conj$ into a substitution
(this is possible due to~\cref{prop:unique}),
apply to the three terms on the right-hand side of $\vdash$,
and then apply the normal conversion check under $\Psi;\Gamma$.
The idea in~\cref{fig:conv-cofib} can be extended in the following ways:
\begin{itemize}
\item To do a normal type-check under conjunctions,
e.g. \fbox{$\Psi;\Gamma,\conj\vdash u:A$}, we apply the substitution to $A$
and run the normal type-check.
\item To check anything that results in a \textit{yes} or \textit{no},
we may check them under a disjunction cofibration $\disj$
by iterating the conjunctions in $\disj$ and run the check under that conjunction.
We return \textit{yes} only when all these checks return \textit{yes},
and return \textit{no} otherwise.
\item To check if two cofibrations $\disj_0,\disj_1$ are equivalent.
We simply check if they satisfy each other. So, $\vdash \disj_0\equiv\disj_1$
is equivalent to $\disj_0\vdash \disj_1\equiv\top$ and $\disj_1\vdash \disj_0\equiv\top$.
\end{itemize}
These ideas give rise to the judgments in~\cref{fig:tyck-w-cofib}.
\begin{figure}[h!]
\begin{align*}
\Psi;\Gamma,\disj&\vdash u\equiv v:A\\
\Psi;\Gamma,\conj&\vdash u:A\\
\Psi;\Gamma,\disj&\vdash \isType{A\equiv B}\\
\Psi;\Gamma,\conj&\vdash \isType{A}\\
\Psi;\Gamma,\disj&\vdash \disj_0\equiv \disj_1
\end{align*}
\caption{Judgments under cofibrations}\label{fig:tyck-w-cofib}
\end{figure}

With all these preparations we can get the typing rule
for partial elements (note that we also add support for $A$ to
reference variables in $\Psi$), as in~\cref{fig:tyck-par}.
\begin{figure}[h!]
\begin{mathpar}
\inferrule{\Psi\vdash \disj\\ \PGvdash\isType{A}}{\PGvdash \isTypeBox{\Partial{\disj}A}}\and
\inferrule{\PGvdash u:A}{\PGvdash \lrbbar u : \Partial\top A}\and
\inferrule{
\forall i\in I.(\Psi\vdash \conj_i)\\
\forall i\in I.(\Psi;\Gamma,\conj_i\vdash u_i:A)\\\\
\forall i,j\in I.(\Psi;\Gamma,\conj_i\land\conj_j\vdash u_i\equiv u_j:A)}
{\PGvdash\lrbbar{\overline{\conj_i\mapsto u_i}}
:\Partial{\left(\bigvee\nolimits_{i\in I} \conj_i\right)}A}
\end{mathpar}
\caption{Typing rule of partial elements}\label{fig:tyck-par}
\end{figure}

\section{Applications of partial elements}

\subsection{Generalized paths}\label{sub:extTy}
In~\cite{CCHM}, the path type is defined using the following rules
(slightly paraphrased for notational consistency):
\begin{mathpar}
\inferrule{\PiGvdash x\isType A\\ \PGvdash a:A[\lcon/x]\\ \PGvdash b:A[\rcon/x]}
{\PGvdash\isTypeBox{\PathTy{\lam x A} a b}}\and
\inferrule{\PiGvdash x u:A}
{\PGvdash \plam x u : \PathTy{\lam x A}{(u[\lcon/x])}{(u[\rcon/x])}}\and
\inferrule{\Hint{\PiGvdash x\isType A}\\\PiGvdash x u:A\\\PGvdash v:\II}{
\PGvdash \papp{(\plam x u)}v \equiv u[v/x]:A[v/x]}\and
\inferrule{\Hint{\PiGvdash x\isType A}\\\PGvdash u:\PathTy{\lam x A} a b}{
\PGvdash \papp u\lcon \equiv a:A[\lcon/x] \\ \text{and} \\ \PGvdash \papp u\rcon \equiv b:A[\rcon/x]}
\end{mathpar}
We may rephrase these rules using the idea of cofibrations and partial elements:
\begin{itemize}
\item The path type \fbox{$\PathTy{\lam x A} a b$}
is similar to a $\Pi$-type \fbox{$(x:\II)\to A$} carrying a partial element.
We denote the whole thing as:
\[\ExtTy{x}{A}{rll}
{x&=\lcon&\mapsto a\\
x&=\rcon&\mapsto b}\]
\item The introduction rule takes a type
\fbox{$\extTy{x}{A}{\overline{\conj_i\mapsto v_i}}$} and checks the following:
\begin{itemize}
\item \fbox{$\PiGvdash x u:A$}, just like checking \fbox{$\lam x u:(x:\II)\to A$}.
\item \fbox{$\forall i. (\Psi,x:\II;\Gamma,\conj_i\vdash u\equiv v_i:A)$},
like saying ``$u$ \textit{matches} every given face''.
\end{itemize}
\item The elimination rule is the same as $\Pi$-types.
\item The computation rules combine the computation rule of $\Pi$-types
and the reduction of partial elements (defined in~\cref{sub:red-cofib}).
\end{itemize}
In particular, we can relax the formation rule to allow multiple intervals
and an arbitrary partial element (but the cofibrations may only use the
provided intervals), and the rest of the rules will still work.
This gives us a better definition of the path type,
which is more convenient to work with.
We may specify only one endpoint (unlike a traditional path,
which always requires two endpoints) of a generalized path as in~\cref{ex:gconcat},
or specify the inner boundaries of squares as in~\cref{ex:gsquare}.

\begin{terminology}
The original idea of generalized paths was introduced in~\cite[\S 2.2 and Fig. 4]{InfCat},
which uses the name ``extension types''. It is also implemented in the proof assistants
\texttt{\textcolor{red}{red}tt}\footnote{\url{https://github.com/RedPRL/redtt}} and
\texttt{\textcolor{blue}{cool}tt}\footnote{\url{https://github.com/RedPRL/cooltt}}.
\end{terminology}

We extend the syntax with generalized paths in~\cref{fig:path}.
\begin{figure}[h!]
\[u,A::= \extTy{\overline{x}}{A}{\overline{\face}}
\mid \plam x u \mid \papp u v
\mid \cdots~\text{(see~\cref{fig:parTy})}\]
\caption{Syntax of generalized paths}\label{fig:path}
\end{figure}

The typing rules are defined in~\cref{fig:tyck-path}.
Note that we make use of reduction on partial elements (\cref{sub:red-cofib}).
\begin{figure}[h!]
\begin{mathpar}
% Formation
\inferrule{\Psi\overline{,x:\II};\Gvdash \isType{A}\\
\forall i.(\overline{x:\II}\vdash\conj_i) \\\\
\PGvdash\lrbbar{\overline{\conj_i\mapsto u_i}}
:\Partial{\left(\bigvee\nolimits_{i} \conj_i\right)}A}
{\PGvdash\isTypeBox{\extTy{\overline{x}}{A}{\overline{\conj_i\mapsto u_i}}}}
\and
% Introduction
\inferrule{\Psi\overline{,x_i:\II};\Gamma\vdash v:A[\overline{x_i/y_i}]\\\\
\forall j. (\Psi\overline{,x_i:\II};\Gamma,\conj_j[\overline{x_i/y_i}]\vdash v\equiv u_j[\overline{x_i/y_i}]:A[\overline{x_i/y_i}])}
{\PGvdash \plam{\overline{x_i}}{v}:\extTy{\overline{y_i}}{A}{\overline{\conj_j\mapsto u_j}}}
\and
% Elimination
\inferrule{\PGvdash \overline {u_i:\II}\\\\
\PGvdash v:\extTy{\overline{x_i}}{A}{\overline\face}}
{\PGvdash \papp v {\overline{u_i}} : A[\overline{u_i/x_i}]}
\and
% Computation
\inferrule{\PGvdash \overline {u_i:\II}\\
\Psi\overline{,x_i:\II};\Gamma,\vdash v:A}
{\PGvdash \papp{(\plam{\overline{x_i}}v)}{\overline{u_i}}\equiv
v[\overline{u_i/x_i}] : A[\overline{u_i/x_i}]}
\and
\inferrule{\PGvdash \overline {u_i:\II}\\
\PGvdash v:\extTy{\overline{x_i}}{A}{\overline\face}\\\\
\PGvdash \lrbbar{\overline\face}[\overline{u_i/x_i}]\equiv \lrbbar{v_0} : A[\overline{u_i/x_i}]}
{\PGvdash \papp v {\overline{u_i}}\equiv v_0 : A[\overline{u_i/x_i}]}
\end{mathpar}
\caption{Typing rules of generalized paths}\label{fig:tyck-path}
\end{figure}

\begin{remark}
With named bindings, the following criterion from~\cref{fig:tyck-path}:
\[\Psi\overline{,x_i:\II};\Gamma,\conj_j[\overline{x_i/y_i}]\vdash v\equiv u_j[\overline{x_i/y_i}]:A[\overline{x_i/y_i}]\]
has an alternatively implementation:
\[\Psi\overline{,y_i:\II};\Gamma,\conj_j\vdash v[\overline{y_i/x_i}]\equiv u_j:A\]
The idea is that we may substitute the term to use the bindings in the type and do the conversion checks,
instead of substituting the type to the use the bindings in the term.
Note that this difference does not exist with index-based bindings.

The alternative implementation is shorter, but the philosophy behind it is very bad:
it is usually preferred to adapt the type during the type checking of a term,
not the other way around. We may want to store a proof of the fact that the term is well-typed,
and it's better if the proof is about the term itself, not a substituted version of it.
During the development of \GuestName, the alternative way was first used and then replaced by the
philosophically more faithful way.
\end{remark}

\begin{example}\label{ex:gconcat}
The path concatenation operation can have a simpler type signature.
With the path type, its type is:
\[\textsf{concat}~(a:A)~(b:A)~(c:A)~(p:\PathTy{\lam \_ A} a b)
~(q:\PathTy{\lam \_ A} b c):\PathTy{\lam \_ A} a c\]
The first three parameters can be inferred from the last two,
but we have to quantify over them anyway.
Using the generalized path type, we may simplify it as:
\begin{align*}
\textsf{concat}~&(p:\extTy{\_}{A}{})~(q:\extTy{\_}{A}{x=\lcon\mapsto p~@~\rcon})\\
&:{\ExtTy{\_}{A}{rll}{x&=\lcon&\mapsto p~@~\lcon\\
x&=\rcon&\mapsto q~@~\rcon}}
\end{align*}
It becomes longer, but the new definition has fewer parameters.
The new definition can be more friendly to program synthesizers
or programming languages that do not support implicit arguments.
\end{example}

\begin{example}\label{ex:gsquare}
Consider \fbox{$p_i:\PathTy{\lam\_ A} \bullet \bullet$} for $i\in\set{0,1,2,3}$
where \fbox{$\bullet:A$} and the following square \fbox{$x:\II,y:\II\vdash u$}:
\carloCTikZ{\carloXy
\carloSqBullets
\fill [pattern color=lightgray,pattern=north west lines] (0,0) rectangle (1,1) ;
\draw[->] (00) -- (01) node [midway,left] {$\papp{p_2}y$} ;
\draw[->] (10) -- (11) node [midway,right] {$\papp{p_3}y$} ;
\draw[->] (00) -- (10) node [midway,above] {$\papp{p_0}x$} ;
\draw[->] (01) -- (11) node [midway,below] {$\papp{p_1}x$} ;
\node (center) at (0.5, 0.5) {$u$} ;}
Traditionally, $u$ has the following type:
\[\PathTy{\lam y{(\PathTy{(\lam\_ A)}{(\papp{p_2}y)}{(\papp{p_3}y)})}}{p_0}{p_1}\]
The generalized version is apparently more straightforward:
\[\ExtTy{x,y}{A}{rll}{
x&=\lcon&\mapsto \papp{p_2}y\\
x&=\rcon&\mapsto \papp{p_3}y\\
y&=\lcon&\mapsto \papp{p_0}x\\
y&=\rcon&\mapsto \papp{p_1}x}\]
\end{example}

\subsection{Cubical subtypes}\label{sub:cubsub}
Cofibrations in generalized paths are always \textit{bounded}
in the sense that they cannot reference arbitrary local variables.
We add the \textit{open} version of generalized paths,
called \textit{cubical subtypes}, into our type theory.
The syntax is defined in~\cref{fig:subtype}.
\begin{figure}[h!]
\begin{align*}
	u,A::= &~~ \subTy{A}{\disj}{\overline\face}\\
\mid &~~ \outS\disj u \mid \inS\disj u\\
\mid &~~ \cdots~\text{(see~\cref{fig:path})}
\end{align*}
\caption{Syntax of cubical subtypes}\label{fig:subtype}
\end{figure}

The type \fbox{$\subTy{A}{\disj}{\overline\face}$} has three parameters: the \textit{ambient} type $A$,
a cofibration $\disj$, and a set of faces ${\overline\face}$ such that:
\[\lrbbar{\overline\face}:\Partial{\disj}A\]
Then, if \fbox{$\PGvdash u:A$} and $u$ \textit{matches} all the given faces
(``match'' as in introduction of generalized paths), then:
\[\inS\disj u:\subTy{A}{\disj}{\overline\face}\]
This gives the introduction rule.
The elimination rule is just the obvious inverse of the introduction rule,
denoted \outSLong.
Hence the typing rules as in~\cref{fig:typing-subtype}.

\begin{figure}[h!]
\begin{mathpar}
\inferrule{\Hint{\PGvdash \isType A} \\
\PGvdash \lrbbar{\overline\face}:\Partial{\disj}A}
{\PGvdash\isTypeBox{\subTy{A}{\disj}{\overline\face}}}\and
\inferrule{\forall i.(\Psi;\Gamma,\conj_i\vdash u\equiv v_i)}
{\PGvdash\inS\disj u:\subTy{A}{\disj}{\overline{\conj_i\mapsto v_i}}}\and
\inferrule{\PGvdash u:\subTy{A}{\disj}{\overline\face}}
{\PGvdash\outS\disj u:A}
\end{mathpar}
\caption{Typing rules of cubical subtypes, typing part}\label{fig:typing-subtype}
\end{figure}

Cubical subtypes compute in the following cases:
\begin{itemize}
\item The elimination rule always cancels the introduction rule.
\item The introduction rule cancels the elimination rule when
they share the same cofibration.
\item The elimination rule computes according to the partial element.
\end{itemize}
Hence the computation rules as in~\cref{fig:compute-subtype}.
\begin{figure}[h!]
\begin{mathpar}
\inferrule{\PGvdash u:A \\ \PGvdash \disj_0 \\ \PGvdash \disj_1}
{\PGvdash \outS{\disj_0}{\inS{\disj_1} u}\equiv u:A} \and
\inferrule{\PGvdash u:\subTy{A}{\disj}{\overline\face}}
{\PGvdash \inS{\disj}{\outS{\disj} u}\equiv u:\subTy{A}{\disj}{\overline\face}}\and
\inferrule{\PGvdash u:\subTy{A}{\disj}{\overline{\conj_i\mapsto v_i}}}
{\forall i.(\Psi;\Gamma,\conj_i\vdash \outS{\disj}u\equiv v_i:A)}
\end{mathpar}
\caption{Typing rules of cubical subtypes, computation part}\label{fig:compute-subtype}
\end{figure}

The set of instances of this cubical subtype is similar\footnote
{It would be nice to say \textit{canonically isomorphic to} here, but the word
\textit{canonical} is being criticized on social media recently,
so it's avoided.} to a subset of the instances of $A$
such that these instances match the given faces.

\begin{remark}
Ideally, we would define cubical subtypes to be \textit{subtypes}
in the sense of subtyping, instead of being explicitly coerced.
However, subtyping in dependent type theories is extremely
sophisticated and may have bad consequences.
\end{remark}

Cubical subtypes, partial elements, and generalized paths are all
based on the idea of cofibrations. Here's a brief table for comparison:

\begin{center}
\begin{tabular}{|l|c|c|c|}
 \hline
 & \textsf{Sub} & \textsf{Partial} & \textsf{Ext} \\ \hline
 Have faces in type? & \checkmark & $\times$ & \checkmark \\ \hline
 Have faces in terms? & $\times$ & \checkmark & $\times$ \\ \hline
 Scope of cofibrations & Open & Open & Bound \\ \hline
%  Number of faces & $1$ & $\geq 1$ & $\geq 0$ \\ \hline
%  Support composition & $\times$ & $\times$ & \checkmark \\ \hline
%  Support transport & $\times$ & $\times$ & \checkmark \\ \hline
\end{tabular}
\end{center}

Cubical subtypes will be useful later in~\cref{sub:transp}.

\section{De Morgan structure}
This section introduces the notion of \textit{transportation rules} and
motivates the De Morgan structure on the interval type.

\subsection{Minimum and maximum squares}\label{sub:minmax}
Consider a path \fbox{$p:\PathTy{\lam\_ A} a b$},
it makes geometric sense to have some path between $p$ and the identity path
on either $a$ or $b$, because they are all connected together.
In \CTT{} we introduce the operators $\lor$ and $\land$,
which are binary operators that return the maximum and minimum of the operands,
respectively. In~\cite[\S 3]{CCHM}, this construction is described as a
``free distributive lattice generated by symbols''.

The rules of $\lor$ and $\land$ are very simple, but listed in~\cref{fig:minmax-rules}.
\begin{figure}[h!]
\begin{mathpar}
\inferrule{\PGvdash u:\II \\\\ \PGvdash v:\II}{\PGvdash u\land v:\II}\and
\inferrule{\PGvdash u:\II}{\PGvdash u\land \lcon\equiv \lcon:\II}\and
\inferrule{\PGvdash u:\II}{\PGvdash u\land \rcon\equiv u:\II}\and
\inferrule{\PGvdash u:\II \\\\ \PGvdash v:\II}{\PGvdash u\lor v:\II}\and
\inferrule{\PGvdash u:\II}{\PGvdash u\lor \lcon\equiv u:\II}\and
\inferrule{\PGvdash u:\II}{\PGvdash u\lor \rcon\equiv \rcon:\II}\and
\inferrule{\PGvdash u:\II}{\PGvdash u\lor u\equiv u:\II}\and
\inferrule{\PGvdash u:\II}{\PGvdash u\land u\equiv u:\II}
\end{mathpar}
\caption{Typing and computation rules of $\land$ and $\lor$}
\label{fig:minmax-rules}
\end{figure}

With these operators we may construct the following squares:
\newcommand{\minmaxPoints}[3]{
\node (00) at (0, 0) {a};
\node (11) at (1.2, 1.2) {b};
\node (10) at (1.2, 0) {#1};
\node (01) at (0, 1.2) {#1};
\fill[pattern color=lightgray,pattern=horizontal lines]
  (00.center) -- (11.center) -- (#2.center) -- cycle ;
\fill[pattern color=lightgray,pattern=vertical lines]
  (00.center) -- (11.center) -- (#3.center) -- cycle ;
}
\newcommand{\minmaxCenter}[2]{
\node[rotate=-45] (center) at (0.6, 0.6) {$\papp p{(#1)}$} ;
\node (text) at (0.6, 1.6) {A ``#2'' square} ;
}
\carloCTikZ{\carloXy
\minmaxPoints{b}{01}{10}
\draw[->] (00) -- (01) node [midway,left] {$\papp p y$} ;
\draw[->] (10) -- (11) node [midway,right] {$b$} ;
\draw[->] (00) -- (10) node [midway,above] {$\papp p x$} ;
\draw[->] (01) -- (11) node [midway,below] {$b$} ;
\minmaxCenter{x\lor y}{maximum}
\shiftTikZ{2.5, 0}{
\minmaxPoints{a}{10}{01}
\draw[->] (00) -- (01) node [midway,left] {$a$} ;
\draw[->] (10) -- (11) node [midway,right] {$\papp p y$} ;
\draw[->] (00) -- (10) node [midway,above] {$a$} ;
\draw[->] (01) -- (11) node [midway,below] {$\papp p x$} ;
\minmaxCenter{x\land y}{minimum}
}}

We also introduce the $\neg$ operator, which is a unary operator that returns
the point of symmetry on $\II$, hence the De Morgan laws are applicable to $\neg$,
$\lor$, and $\land$. The rules are like:
\begin{align*}
	\neg(u\land v)&\equiv \neg u \lor \neg v\\
	\neg(u\lor v)&\equiv \neg u \land \neg v\\
\end{align*}
Some other obvious rules on $\neg$ include $\neg(\neg u)\equiv u$,
$\neg\lcon\equiv\rcon$, and $\neg\rcon\equiv\lcon$.

\begin{exercise}
Think about the squares we can construct with a path
$p:\PathTy{\lam\_A}a b$ and the De Morgan structures on $\II$,
like in~\cref{rem:sq-ori}.
\end{exercise}

\begin{exercise}\label{ex:norm-i}
Show that terms of type $\II$ (generated by variables, $\neg$, $\land$, and $\lor$)
strongly normalizes to unique normal forms.
This implies that expressions of type $\II$ in open contexts has a
decidable conversion check algorithm.
\end{exercise}

Note that there are established facts that support~\cref{ex:norm-i},
so it is not recommended to do it in a brute-force way.

\begin{lem}
There exists an isomorphism between expressions of type $\II$ in open contexts
and cofibrations in the same context.
\end{lem}
\begin{proof}
By constructing the following bijective function.
\[\begin{array}{rl}
\phi(x)&:=x=\rcon\\
\phi(\neg x)&:=x=\lcon\\
\phi(x\land y)&:=\phi(x)\land\phi(y)\\
\phi(x\lor y)&:=\phi(x)\lor\phi(y)
\end{array}\]
It is left to the reader to construct its inverse.
\end{proof}

\subsection{Transportation rules}\label{sub:transp}
\begin{defn}[Freeze]\label{defn:freeze}
A type-line $A:\II\to\UU$ \textit{freezes} on a cofibration $\disj$ if
it is \textit{constant} under $\disj$. In terms of typing judgment:
\[\Psi,x:\II;\Gamma,\disj\vdash \isTypeBox{A~\lcon\equiv A~x}\]
The judgment is from~\cite[\S 2.1]{HCompPDF}.
\end{defn}
Here is an illustration of~\cref{defn:freeze} inspired from an online cubical tutorial
by Favonia. Consider the type-line $A:\II\to\UU$, how can it be frozen under a cofibration?
\begin{enumerate}
\item If $A$ is already constant,
then it is frozen under all cofibrations, which is not interesting.
\item Consider a non-constant $A$ which is constant under $\disj$.
For convenience, let $A:=\plam x{A_0}$. Apparently,
there are subexpressions in $A_0$ involving the interval variable $x$.
\item Recall~\cref{sub:tyck-cofib}, this means under some interval substitution,
these subexpressions will reduce and will no longer refer to $x$.
\item So, these subexpressions have to refer to interval variables other than $x$
which are substituted away under $\disj$!
\end{enumerate}
So, suppose $A_0$ also refers to $y:\II$. This makes $A$ a square:
\carloCTikZ{\carloXy
\carloSqBullets
\draw[->] (00) -- (01) node [midway,left] {$A~\lcon$} ;
\draw[->] (10) -- (11) node [midway,right] {$A~\rcon$} ;
\draw[->] (00) -- (10) node [midway,above] {$A_0[\lcon/y]$} ;
\draw[->] (01) -- (11) node [midway,below] {$A_0[\rcon/y]$} ;}
Since $\disj$ substitutes $y$ away, either $y=\lcon\in\disj$ or $y=\rcon\in\disj$.
We may choose either, say, $y=\lcon$. We may also forget about the other conditions,
as they are unrelated to the square. Then, since $A$ \textit{freezes} under $y=\lcon$,
the corresponding face must be constant, resulting in a triangle:
\carloCTikZ{\carloXy
\carloSqBullets
\draw[->] (00) -- (01) node [midway,left] {$A~\lcon$} ;
\draw[->] (10) -- (11) node [midway,right] {$A~\rcon$} ;
\draw[equals arrow] (00) -- (10) node [midway,above] {} ;
\draw[->] (01) -- (11) node [midway,below] {$A_0[\rcon/y]$} ;
\shiftTikZ{2, 0}{
\carloSqVertices{~~}
\draw[->] (10) -- (01) node [midway,left] {$A~\lcon$} ;
\draw[->] (10) -- (11) node [midway,right] {$A~\rcon$} ;
\draw[->] (01) -- (11) node [midway,below] {$A_0[\rcon/y]$} ;
}}
\begin{remark}
Note that the judgment in~\cref{defn:freeze} is not equivalent to:
\[\Psi;\Gamma,\disj\vdash \isTypeBox{A~\lcon\equiv A~\rcon}\]
Because a nontrivial loop (hence the two sides are the same,
but it's distinguishable from the identity path, hence not constant)
satisfies the latter condition.
\end{remark}
In \CTT, for every type-line \fbox{$A:\II\to\UU$},
we add a function $\coeLong:A~\lcon\to A~\rcon$ to the type theory,
satisfying the following properties:
\begin{itemize}
\item It is an isomorphism -- so we want
an inverse $\coeLong_A^{-1}:A~\rcon\to A~\lcon$.
One may easily guess that this is achieved by the
$\coeLong$ function on type $\plam x A~(\neg x)$.
\item There exists a cofibration $\disj$ that $A$ freezes (\cref{defn:freeze}) on it.
Furthermore, we want that for every $u:A~\lcon$:
\[\disj\vdash \coeLong(u)\equiv u:A~\lcon\]
\item It computes on type formers, e.g. for the type line $\lam x{A~x\to B~x}$, we have:
\[\coeLong:(A~\lcon\to B~\lcon)\to(A~\rcon\to B~\rcon)\]
And we expect this function to behave like:
\begin{align*}
&\text{take}~f:A~\lcon\to B~\lcon~\text{and}~a:A~\rcon&&\text{from input}\\
\implies &\coeLong^{-1}(a):A~\lcon&&\text{by type line}~\lam x{A~x}\\
\implies &f(\coeLong^{-1}(a)):B~\lcon&&\text{by function application}\\
\implies &\coeLong(f(\coeLong^{-1}(a))):B~\rcon&&\text{by type line}~\lam x{B~x}\\
\end{align*}
Hence the well-typed result: an instance of type $B~\rcon$.
This means the function does not obviously break canonicity.
\end{itemize}
The behavior of $\coeLong$ is considered an extra set of \textit{rules}
called \textit{transportation rules} on each type former,
just like introduction rules, elimination rules,
and computation rules.

To achieve all the properties we have asked for,
we add the following ``function'' to \CTT:

\begin{figure}[h!]
\[u,A::= \coe{A}{\disj} \mid \cdots~\text{(see~\cref{fig:subtype})}\]
\caption{Syntax of transport}
\label{fig:syntax-transp}
\end{figure}

It is actually not the normal sense of function,
because the function body can't be expressed internally in the type theory.
It is more like an operator whose behavior is induced by the type $A$.

The implementation is basically a rewrite of~\cite[\S 2.1, \S 3]{HCompPDF},
as in~\cref{fig:typing-transp}.

\begin{figure}[h!]
\begin{mathpar}
\inferrule{\PGvdash A:\II\to\UU \\
\Psi\vdash\overline{\conj_i} \\ \PiGvdash{x}\isTypeBox{A~x\equiv A~\lcon}}{
\coe{A}{\overline{\conj_i}}:(u:A~\lcon) \to
 \subTy{(A~\rcon)}{\overline{\conj_i}}{\overline{\conj_i\mapsto u}}}
\end{mathpar}
\caption{Typing of the transportation rule}
\label{fig:typing-transp}
\end{figure}

The type signature is a bit complicated because of the cubical subtype
used to express the redution of $\coeLong$. We may simplify by separating the
computation behavior from the typing, as in~\cref{fig:typing-transp-2}.
This approach does not have the cubical subtype boilerplate, which leads to cleaner code.

\begin{figure}[h!]
\begin{mathpar}
\inferrule{\PGvdash A:\II\to\UU \\
\Psi\vdash\disj \\ \PiGvdash{x}\isTypeBox{A~x\equiv A~\lcon}}{
\PGvdash\coe{A}{\disj}:A~\lcon \to A~\rcon}\and
\inferrule{\PGvdash A:\II\to\UU \\ \Psi\vdash\disj}{
\Psi;\Gamma,\disj\vdash\fbox{$\coe{A}{\disj}\equiv \lam x x$}:
A~\lcon \to A~\lcon\footnotemark}
\end{mathpar}
\caption{Alternative way to type the transportation rule}
\label{fig:typing-transp-2}
\end{figure}
\footnotetext{Note that the conversion rule is well-typed because under $\disj$, $A~\lcon\equiv A~\rcon$}

\begin{remark}
The work~\cite{CubicalAgda} has implemented it as a built-in function
with the following type signature:
\[\coeLong:(A:\II\to\mathbb{U})~(\psi:\II)~(A~\lcon)\to(A~\rcon)\]
However, invocations to this function is valid only when $A$ is constant under $\psi$,
and this criterion is not expressed in the type signature.
It is said to be a design flaw\footnote{\url{https://github.com/agda/agda/issues/3509\#issuecomment-456201777}}
in the Agda bug tracker.
\end{remark}

\begin{lem}[Fill]\label{lem:transpFill}
Given a well-typed term: \[\coe A{\disj}{(u)}\]
There exists the following path: \[p:\ExtTy x {A~x}{ccl}{
	x&=\rcon&\mapsto \coe A{\disj}{(u)} \\
	x&=\lcon&\mapsto u}\]
In more traditional notations:
\[p:\PathTy{A} u {(\coe A{\disj}{(u)})}\]
Diagrammatically:
\[\begin{tikzcd}
	u && {\coe A{\disj}{(u)}}
	\arrow["p"', from=1-1, to=1-3]
\end{tikzcd}\]
\end{lem}
\begin{proof}
Let
\[p:=\plam x{\coe{\lam y{A~(x\land y)}}
 {\disj\lor x=\lcon}(u)}\]
It is left to the reader to check the well-typedness of these constructions
(see discussion in~\cite{HCompPDF} which is extremely brief but still useful).
In some sense, this construction motivates the De Morgan structure on $\II$.
\end{proof}

The transportation rules for type formers are omitted as they are
very long and are already written in~\cite[\S 3]{HCompPDF}. By comparing~\cref{lem:transpFill}
and~\cite[\S 2.1]{HCompPDF}, one should be able to translate the rest.

\subsection{Homogeneous composition}
There is another primitive construction which is needed to define the
transportation rules for the (generalized) path type:
the \textit{homogeneous composition} operation.

\TODO